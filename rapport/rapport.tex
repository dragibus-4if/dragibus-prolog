\documentclass[a4paper, 11pt]{article}

% Paquets
\usepackage[utf8]{inputenc}
\usepackage[frenchb]{babel}
\usepackage[T1]{fontenc}
\usepackage{textcomp}
\usepackage{amsmath,amssymb}
\usepackage{lmodern}
\usepackage[top = 1.5cm, bottom = 2cm, left=2cm, right=2cm]{geometry}
\usepackage{graphicx}
\usepackage{xcolor}
\usepackage{microtype}
\usepackage{listings}
\usepackage{hyperref}

% Commandes
\newcommand{\HRule}{\rule{\linewidth}{0.5mm}}

% Configuration des listings (code source)
\definecolor{gray}{rgb}{0.7, 0.7, 0.7}
\lstset{
    language=Prolog,
    % basicstyle=\footnotesize,
    basicstyle=\ttfamily\small, %
    identifierstyle=\color{red!50!black!100}, %
    keywordstyle=\color{blue}, %
    stringstyle=\color{black!60}, %
    commentstyle=\color{green!65!black!95}, %
    % backgroundcolor=\color{white},
    % commentstyle=\itshape\color{gray},
    % keywordstyle=\bf\color{black},
    % stringstyle=\color{green},
    breakatwhitespace=false,
    breaklines=true,
    deletekeywords={concat},
    keepspaces=true,
    showspaces=false,
    showstringspaces=false,
    showtabs=false,
    extendedchars=true, %
    columns=flexible, %
    numbers=left, %
    numberstyle=\tiny, %
    captionpos=b,
    breakautoindent=true, %
    tabsize=2
}

\definecolor{Zgris}{rgb}{0.87,0.85,0.85}

\newsavebox{\BBbox}
\newenvironment{DDbox}[1]{
\begin{lrbox}{\BBbox}\begin{minipage}{\linewidth}}
{\end{minipage}\end{lrbox}\noindent\colorbox{Zgris}{\usebox{\BBbox}} \\
[.5cm]}

% Document
\begin{document}
% Page de garde
\begin{titlepage}
    \begin{center}
        % Titre
        \HRule \\[0.4cm]
        {\huge \bfseries Projet Prolog 4IF}
        \HRule \\[1.5cm]

        % Auteurs
        \begin{minipage}{0.8\textwidth}
            \center
            \large
            Hexanôme 4104 - Equipe Dragibus
        \end{minipage}

        \vfill
        % Fin de page
        {\large \today}
    \end{center}
\end{titlepage}

% \tableofcontents
% \newpage

\section{Généalogie}
On crééra une base de connaissances en spécifiant les relations "parent"
directes. parent(X, Y) étant vrai si X est le parent de Y. \\
\begin{DDbox}{\linewidth}
    \lstinputlisting{../exercice/genealogie/parent.pl}
\end{DDbox}

Définition du lien grand-parent par une propagation de la relation parent sur 2
niveaux. grandParent(X, Y) est vrai si X est le grand parent de Y. \\
\begin{DDbox}{\linewidth}
    \lstinputlisting{../exercice/genealogie/grandParent.pl}
\end{DDbox}

Définition du lien ancêtre par une propagation récursive sur plusieurs
niveaux, jusqu'à satisfaction. ancetre(X, Y) est vrai si X est un ancetre de Y. \\
\begin{DDbox}{\linewidth}
    \lstinputlisting{../exercice/genealogie/ancetre.pl}
\end{DDbox}

Définition de la liste des ancêtres à partir d'un élément de la
famille.  On utilise setof car cette partie n'est pas sur les
listes, et son utilisation est pratique pour éliminer les
doublons dans les candidats à la relation ancêtre. On définie de
la meme façon pour récupérer la liste des descendants. \\
ancetres(X, L) est vrai si L est l'ensemble des ancetres de X. \\
descendants(X, L) est vrai si L est l'ensemble des descendants de X. \\
Dans ces deux prédicats, X ne peut pas être déterminée à partir
de L, et donc ne peut pas être une variable. \\
\begin{DDbox}{\linewidth}
    \lstinputlisting{../exercice/genealogie/ancetres.pl}
\end{DDbox}

Définition de la relation frère-soeur complète en partant du
constat que celle-ci correspond au partage des parents. Cette
relation n'est pas réflexive, une personne ne pouvant pas être
son propre frère-soeur. frereSoeur(X, Y) est vrai si X et Y ont
le meme parent (donc sont frère-soeur). \\
\begin{DDbox}{\linewidth}
    \lstinputlisting{../exercice/genealogie/frereSoeur.pl}
\end{DDbox}

Définition de la relation oncle-tante par une relation
frère-soeur pour un parent. oncleTante(X, Y) est vrai si X est
l'oncle (ou la tante) de Y. \\
\begin{DDbox}{\linewidth}
    \lstinputlisting{../exercice/genealogie/oncleTante.pl}
\end{DDbox}

Définition de la relation cousin par une relation frère-sœur
entre deux parents.  cousin(X, Y) est vrai si X et Y sont cousin
(càd que leur parent sont freres). \\
\begin{DDbox}{\linewidth}
    \lstinputlisting{../exercice/genealogie/cousin.pl}
\end{DDbox}

% \subsubsection{Tests}
% \begin{DDbox}{\linewidth}
%     \lstinputlisting{../tests/genealogie.pl}
% \end{DDbox}

\newpage
\section{Listes}
element(X, L, LprivX) est vrai si L/X est une liste composé des
éléments de L privé une fois de X. L'ordre doit etre préservé.  Si
X n'est pas dans L, alors le prédicat est faux.  Les trois
paramètres peuvent etre des variables. \\
\begin{DDbox}{\linewidth}
    \lstinputlisting{../exercice/liste/element.pl}
\end{DDbox}

extract(L1, L2) est vrai si L2 est un sous ensemble de L1. Les
deux paramètres peuvent etre des variables. \\
\begin{DDbox}{\linewidth}
    \lstinputlisting{../exercice/liste/extract.pl}
\end{DDbox}

concat(L1, L2, L1nL2) est vrai si L1nL2 est la concaténation de L1
et L2. \\
\begin{DDbox}{\linewidth}
    \lstinputlisting{../exercice/liste/concat.pl}
\end{DDbox}

inv(L1, L2) est vrai si L1 est l'inverse de L2. On remarque que si
L1 n'est pas une variable et que L2 en est une, le prédicat ne se
termine pas. \\
\begin{DDbox}{\linewidth}
    \lstinputlisting{../exercice/liste/inv.pl}
\end{DDbox}

subsAll(E, X, L1, L2) est vrai si L2 est la liste L1 avec les
éléments E remplacés par des X. Dans le cas ou E est une variable,
il prend la première valeur de L1. \\
\begin{DDbox}{\linewidth}
    \lstinputlisting{../exercice/liste/subsAll.pl}
\end{DDbox}

% \subsubsection{Tests}
% \begin{DDbox}{\linewidth}
%     \lstinputlisting{../tests/liste.pl}
% \end{DDbox}

\newpage
\section{Arithmetique}
element(Idx, X, L) est vrai si X est à la position Idx dans L (en
commençant à compter à partir de 1). On doit passer par une
fonction element/4 pour intégrer un accumulateur en paramètre. \\
\begin{DDbox}{\linewidth}
    \lstinputlisting{../exercice/arithmetique/element.pl}
\end{DDbox}

% \subsubsection{Tests}
% \begin{DDbox}{\linewidth}
%     \lstinputlisting{../tests/arithmetique.pl}
% \end{DDbox}

\newpage
\section{Ensembles}
list2ens(L, E) est vrai quand E correspond a l'ensemble des
elements de L, sans doublon. L'ordre de la liste initiale est
conserve.  On doit alors définir un predicat list2ens/3 pour
utiliser un accumulateur en paramètre. Egalement, on définit
simplement un prédicat element/2 qui est vrai lorsqu'un élément X
est dans une liste L. \\
\begin{DDbox}{\linewidth}
    \lstinputlisting{../exercice/ensemble/list2ens.pl}
\end{DDbox}

ensemble(L) est vrai quand L est un ensemble, et donc ne contient
pas de doublons. On réutilise le prédicat précédemment fait
list2ens. \\
\begin{DDbox}{\linewidth}
    \lstinputlisting{../exercice/ensemble/ensemble.pl}
\end{DDbox}

% \subsection{Test}
% \begin{DDbox}{\linewidth}
%     \lstinputlisting{../tests/ensemble.pl}
% \end{DDbox}

\end{document}
