\documentclass{article}

% Paquets
\usepackage[utf8]{inputenc}
\usepackage{listings}
\usepackage{color}

% Commandes
\newcommand{\HRule}{\rule{\linewidth}{0.5mm}}

% Configuration des listings (code source)
\definecolor{gray}{rgb}{0.7, 0.7, 0.7}
\lstset{
    basicstyle=\footnotesize,
    backgroundcolor=\color{white},
    commentstyle=\itshape\color{gray},
    keywordstyle=\bf\color{black},
    stringstyle=\color{green},
    breakatwhitespace=false,
    breaklines=true,
    language=Prolog,
    deletekeywords={concat},
    keepspaces=true,
    showspaces=false,
    showstringspaces=false,
    showtabs=false,
    tabsize=2
}

% Document
\begin{document}

% Page de garde
\begin{titlepage}
\begin{center}

% Titre
\HRule \\[0.4cm]
{\huge \bfseries Projet Prolog 4IF}
\HRule \\[1.5cm]

% Auteurs
\begin{minipage}{0.8\textwidth}
\center
\large
Hexanôme 4104
\end{minipage}

\vfill

% Fin de page
{\large \today}

\end{center}
\end{titlepage}

\section{Exercices}

\subsection{Généalogie}

\lstinputlisting{../exercice/genealogie.pl}
\lstinputlisting{../tests/genealogie.pl}

\subsection{Listes}
\lstinputlisting{../exercice/liste.pl}
\lstinputlisting{../tests/liste.pl}

\subsection{Arithmetique}
\lstinputlisting{../exercice/arithmetique.pl}
\lstinputlisting{../tests/arithmetique.pl}

\subsection{Ensembles}
\lstinputlisting{../exercice/ensemble.pl}
\lstinputlisting{../tests/ensemble.pl}

\section{Projet}

Nous nous sommes basés sur le jeu du Perudo (Liar's Dice).

\subsection{Règles du jeu}
Chaque joueur a un certain nombre de dés à lancer.  Ces dés sont des dés à six
faces normaux, excepté le 1, qu'on appelle Paco, et qui est une sorte de joker :
il a toutes les autres valeurs à la fois.
\\
Ainsi, par exemple, si on a sur
cinq dés Paco-deux-trois-cinq-cinq, on comptera 2 deux (le vrai et le Paco), 2
trois(le vrai et le paco), 1 quatre (le paco), 3 cinq (le paco et les deux
vrais), ou 1 six(le paco).
\\
Les enchères sur les faces des dés sont toujours au moins. Si on parie sur 5
six, il faut qu'il y ait au moins 5 dés de valeur six pour que l'enchère soit
correcte.

\end{document}
